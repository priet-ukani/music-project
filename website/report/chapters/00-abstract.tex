% Abstract
\chapter*{Abstract}
\addcontentsline{toc}{chapter}{Abstract}

\vspace{0.5cm}

India's musical heritage represents one of the world's most diverse and ancient continuous musical traditions, spanning classical systems, folk genres, tribal music, and devotional forms across its vast geographical and cultural landscape. However, this rich tapestry of musical knowledge faces significant challenges in the modern era, including inadequate documentation, declining patronage for traditional forms, and limited accessibility for learners and researchers.

This project presents \textbf{Musical Map of India}, an interactive digital platform designed to document, preserve, and present India's regional musical diversity through an engaging geographical interface. The platform integrates ethnomusicological research with modern web technologies to create an educational resource that bridges the gap between academic documentation and public accessibility.

The project encompasses comprehensive documentation of musical traditions from over 20 Indian states and union territories, covering distinct musical systems including Hindustani and Carnatic classical traditions, regional folk genres, tribal music, and devotional forms. Each region's musical profile includes detailed information about rhythmic systems (tala), melodic structures (raga), traditional instruments, performance contexts, social and cultural frameworks, and historical influences. The platform features authentic audio samples, interactive soundscape mixing capabilities, instrument galleries, and curated information about featured artists and contemporary musical events.

The research methodology combines ethnomusicological analysis from scholarly sources with digital audio collection using ethical sampling practices. The platform architecture employs React-based interactive mapping, dynamic content rendering, and responsive audio playback systems to ensure an engaging user experience across devices.

Key findings from this work highlight several critical aspects of India's musical landscape: the fundamental distinction between North Indian (Hindustani) and South Indian (Carnatic) classical systems, the remarkable diversity of folk and tribal music that varies significantly even within state boundaries, the crucial role of hereditary musician communities in preserving traditional knowledge, the impact of patronage systems on musical evolution, and the ongoing challenges of modernization and cultural homogenization.

The platform serves multiple audiences: students researching Indian music, educators teaching cultural studies, musicians learning about regional traditions, and the general public interested in India's cultural heritage. By presenting complex musicological concepts through accessible narratives and interactive elements, the project demonstrates how digital technologies can enhance cultural preservation efforts while maintaining ethnomusicological rigor.

This work contributes to the growing field of digital musicology and cultural heritage preservation, offering a model for how interactive, geographically organized platforms can make specialized knowledge accessible to broader audiences. The project also raises important questions about digital preservation ethics, the role of technology in cultural documentation, and the balance between academic accuracy and public engagement.

\vspace{0.5cm}

\textbf{Keywords:} Indian Music, Digital Musicology, Cultural Heritage, Ethnomusicology, Interactive Mapping, Raga, Tala, Regional Music, Cultural Preservation, Web-based Education

\clearpage
